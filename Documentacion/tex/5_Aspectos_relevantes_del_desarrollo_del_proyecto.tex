\chapter{Aspectos relevantes del desarrollo del proyecto}

En un principio se contemplo la posibilidad de realizar el trabajo en Google Colaboratory\footnote{Sitio web de Googe Colab: \url{https://colab.research.google.com/notebooks/intro.ipynb}} pero por problemas de comunicación entre el notebook de Google Colab y la maquina local se terminó descartando la idea.

Posteriormente se intento realizar el trabajo en un notebook local utilizando Jupyter, incluido en el paquete de Anaconda 3, pero debido a la restricción de permisos que tienen los exploradores de internet se terminó descartando también.

Para el interprete de C se pensó aprovechar la herramienta gdb incluida en Linux de base y en Windows mediante MinGW pero ante la imposibilidad de crear una comunicación continua entre la consola de la maquina y la aplicación se terminó por descartar esta opción.

Se tenía la intención de diseñar el parser y lexer enteros utilizando una gramática\footnote{Enlace a gramática; \url{https://www.lysator.liu.se/c/ANSI-C-grammar-y.html}} y un léxico\footnote{Enlace a léxico: \url{https://www.lysator.liu.se/c/ANSI-C-grammar-l.html}} encontrados en internet.