\capitulo{3}{Conceptos teóricos}

\begin{description}
    \item[Compilador]: un compilador es un traductor que se encarga de convertir el código fuente a código máquina para poder ser entendido por una maquina
    
    \item[Interprete]: a diferencia de los compiladores que traducen todo el código de un vez, los interpretes, van convirtiendo el código poco a poco, linea a linea, instrucción a instrucción, etc.
    
    \item[Lexer o analizador léxico]: el lexer analiza una secuencia de caracteres y la convierte en una secuencia de tokens, por ejemplo: a = 2 + 6. El lexer lo traduciría como: 
    \begin{enumerate}
        \item identificador [a]
        \item operador de asignación [=]
        \item constante numérica [2]
        \item operador de suma [+]
        \item constante numérica [6]
    \end{enumerate}
    
    \item[Parser o analizador sintáctico]: agrupa los token resultados del lexer formando frases gramaticales que posteriormente serán analizadas por el compilador. Utilizando el ejemplo visto con el lexer (a = 2 + 6), así quedaría con el parser:
    
    \Tree [.{operación de asignación} {ID a} {op =} [.{operación suma} {num 2} {op +} {num 6} ] ]
\end{description}


\footnote{Definiciones extraídas del libro: Jiménez Millán, José Antonio. (2014). Compiladores y procesadores del lenguaje. Cádiz, España: Servicio de Publicaciones de la Universidad de Cádiz.}

Algunos conceptos teóricos de \LaTeX \footnote{Créditos a los proyectos de Álvaro López Cantero: Configurador de Presupuestos y Roberto Izquierdo Amo: PLQuiz}.


