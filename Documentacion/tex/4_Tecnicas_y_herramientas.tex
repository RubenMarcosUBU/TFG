\chapter{Técnicas y herramientas}

Para la realización de este proyecto se usará un debugger de C y se adaptará a una nueva interfaz mas clara y limpia para los alumnos nuevos que den sus primeros pasos en programación.

Para la realización del proyecto me he ayudado de un parser de C para la librería PLY de Python, pycparser \footnote{Repositorio de GitHub: \url{https://github.com/eliben/pycparser}}

Para programar se ha utilizado Spyder del paquete de Anaconda 3 \footnote{Sitio web de anaconda para su descarga: \url{https://www.anaconda.com}}

Para la documentación en \LaTeX se ha utilizado Overleaf\footnote{Sitio web de Overleaf: \url{https://www.overleaf.com}}, he utilizado Overleaf porque al tener el proyecto en nube me es mas cómodo y fácil acceder a él desde cualquier dispositivo además de incluir un previsualizador online del pdf resultante del documento de \LaTeX

Utilizo el compilador de gcc, que al no venir de base en Windows se ha utilizado el paquete de herramientas de MinGW \footnote{Sitio web de MinGW: \url{http://mingw.org/}}

\tablaSmall{Herramientas y tecnologías utilizadas en cada parte del proyecto}{l c c}{herramientasportipodeuso}
{ \multicolumn{1}{l}{Herramientas} & Aplicación & Memoria \\}{ 
Spyder & X & \\
Pycparser & X & \\
GitHub Desktop & X & X \\
Overleaf &  & X \\
}