\capitulo{2}{Objetivos del proyecto}

El principal objetivo del proyecto es la creación de una interfaz limpia y clara que permita monitorizar las variables para nuevos alumnos que nunca hayan estudiado programación. Para ello nos basaremos en interfaces de desarrollo vistos en distintas clases prácticas: MatLab, Spyder y Eclipse.
Uno de los objetivos es que el interprete de C muestre por pantalla el estado de todas la variables utilizadas en la ejecución, mostrando su valor, su espacio en memoria y su tipo ver Tabla 1.

\caption{Ejemplo de visualización de la tabla de variables}
\label{tab:table1}
\begin{tabular}{|l|c|c|c|}
\hline 
\textbf{Nombre} & \textbf{Tipo} & \textbf{Hex} & \textbf{valor}\\ 
\hline 
Num & int & 0x5F8 & 1528 \\ 
\hline 
Letra & char & 0x3F & ? \\ 
\hline 
\rhd Vector & int[3] & - & - \\
\hline 
\hline 
\bigtriangledown Vector & int[3] & - & - \\

Vector[0] & int & 0x17 & 23 \\ 

Vector[1] & int & 0x39 & 57 \\  
 
Vector[2] & int & 0x8D & 141 \\  
\hline 
\rhd Estructura & Struct & - & - \\
\hline 
\hline 
\bigtriangledown Estructura & Struct & - & - \\

Estructura[0] & int & 0x5D & 93 \\ 

Estructura[1] & char & 0x40 & 57 \\  
 
Estructura[2] & char & 0x20 &   \\
\hline 
\end{tabular} 
Otro de los objetivos es que el interprete nos permita ejecutar las lineas de código una a una
Siguiendo con el objetivo anterior permitir la inclusión de breakpoints para facilitar la depuración de código