\chapter{Objetivos del proyecto}

\begin{itemize}
\item El principal objetivo del proyecto es la creación de una interfaz limpia y clara que permita monitorizar las variables para nuevos alumnos que nunca hayan estudiado programación. Para ello nos basaremos en interfaces de desarrollo vistos en distintas clases prácticas: MatLab, Spyder y Eclipse.
\item Uno de los objetivos es que el interprete de C muestre por pantalla el estado de todas la variables utilizadas en la ejecución, mostrando su valor, su espacio en memoria y su tipo ver \ref{tabla:1}.

\tablaSmall{Ejemplo de visualización de la tabla de variables}{|l|c|c|}{1}{
 \textbf{Nombre} & \textbf{Tipo} & \textbf{valor}\\ 
}{
Num & int & 1528 \\ 
Letra & char &  ? \\ 
\hline
Vector & int[3] & [23,57,31] \\ 
\hline
Libro & Book(Struct) & - \\

->Titulo & char[100] & El viento en los sauces \\ 

->Autor & char[50] & Kenneth Grahame \\  
 
->ISBN & int & 1530059984 \\
}

\item Otro de los objetivos es que el interprete nos permita ejecutar las lineas de código una a una
\item Siguiendo con el objetivo anterior permitir la inclusión de breakpoints para facilitar la depuración de código
\end{itemize}