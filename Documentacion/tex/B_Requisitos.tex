\apendice{Especificación de Requisitos}

\section{Introducción}

El objetivo general de este programa es la depuración de código en lenguaje C. Para ello se requerirá poder abrir, crear, editar, compilar y depurar archivos con la extensión .c

\section{Objetivos generales}

\item OBJ-01:Depuración de código C.

\section{Catalogo de requisitos}

\item RI-01:Abrir archivos .c y .h
\item RI-02:Crear archivos .c y .h
\item RI-03:Editar archivos .c y .h
\item RI-04:Guardar archivos .c y .h
\item RI-05:Compilación de archivos .c
\item RI-06:Depuración de archivos .c

\section{Especificación de requisitos}



\outerTable{01}{Abrir archivos .c y .h}
{Versión 1.0}
{OBJ-01:Depuración de código C.}
{--}
{El programa deberá ser capaz de abrir archivos con la extensión .c y .h}
{El usuario deberá haber iniciado el asistente}
{{1: & solicita al sistema el explorador de archivos \\
\hline
2: & una vez seleccionado el archivo se accederá al editor \\}
{2: & el usuario abrirá un archivo con extensión no permitida \\
\hline
2: & el usuario abrirá un archivo corrupto \\}}
{El usuario deberá tener cargado un archivo con la extensión .c o .h}


\outerTable{02}{Crear archivos .c y .h}
{Versión 1.0}
{OBJ-01:Depuración de código C.}
{--}
{El programa deberá ser capaz de crear nuevos archivos con la extensión .c y .h}
{El usuario deberá haber iniciado el asistente}
{{1: & solicita al programador la ruta en la que quiere trabajar \\
\hline
2: & se solicita al programador el nombre del archivo \\
\hline
3: & se generará un archivo con el nombre y la extensión seleccionados \\}
{
1: & el usuario selecciona una ruta inexistente: se le preguntará sobre si crear las carpetas correspondientes o si quiere seleccionar una nueva ruta \\
\hline
3: & el usuario escogerá un nombre ya existente y se le preguntara si quiere sobreescribir el archivo \\}}
{El sistema deberá haber creado el archivo correspondiente y tenerlo cargado en el asistente}

\outerTable{03}{Editar archivos .c y .h}
{Versión 1.0}
{OBJ-01:Depuración de código C.}
{\item\nameref{RF:01}
\item\nameref{RF:02}}
{El programa deberá ser capaz de editar archivos con la extensión .c y .h}
{El usuario deberá tener cargado un archivo en el asistente}
{{1: & el usuario modificará lineas de código \\}
{
1: & el usuario no dispondrá de permisos de escritura por lo cual no se le permitirá modificar lineas de codigo \\}}
{El asistente deberá tener en memoria el archivo modificado para guardarlo posteriormente}

\outerTable{04}{Guardar archivos .c y .h}
{Versión 1.0}
{OBJ-01:Depuración de código C.}
{\nameref{RF:03}}
{El programa deberá ser capaz de guardar archivos con la extensión .c y .h}
{El usuario deberá tener cargado en el programa un archivo}
{{1: & solicita al sistema el explorador de archivos \\
\hline
2: & se sobrescribirá el archivo con el nombre y la extensión elegidas \\}
{2: & no haya espacio en el disco duro, se vuelve al paso 1 \\
\hline
2: & no existe dicho archivo, el sistema creará uno con el nombre y extensión elegidas \\}}
{El sistema deberá haber modificado correctamente el archivo}

\outerTable{05}{Compilar archivos .c}
{Versión 1.0}
{OBJ-01:Depuración de código C.}
{\nameref{RF:04}}
{El programa deberá ser capaz de compilar archivos .c y crear ejecutables a partir de ellos}
{El usuario deberá haber guardado los cambios del archivo}
{{1: & sobrescribirá el archivo .o correspondiente a dicho archivo \\
\hline
2: & sobrescribirá el ejecutable \\}
{1: & las modificaciones del archivo no han sido guardadas: se ejecutará el RF-04: Guardar archivos .c y .h \\
\hline
1: & el código de un error de compilación y se interrumpirá el proceso \\
\hline
1: & no existe el archivo .o correspondiente por lo que se creará uno nuevo \\
\hline
2: & no existe el ejecutable correspondiente por lo que se creará uno nuevo \\}}
{Disponer de un ejecutable funcional}

\outerTable{06}{Ejecutar linea de código (entrando en función)}
{Versión 1.0}
{OBJ-01:Depuración de código C.}
{\nameref{RF:05}}
{El programa deberá ser capaz de ejecutar una a una las lineas de código de un programa en lenguaje C}
{El usuario deberá haber compilado el archivo}
{{1: & se ejecutará una linea de código \\
\hline
2: & Irá mostrando las variables en memoria de ese programa acorde al \nameref{RF:09}\\}
{1: & no se dispone de un ejecutable funcional: se vuelve al \nameref{RF:05} \\
\hline
2: & encuentra una función por saltará al RF-07 Ejecutar linea de código (entrando en función) sobre esa función\\}}
{--}

\outerTable{07}{Ejecutar linea de código (saltando función)}
{Versión 1.0}
{OBJ-01:Depuración de código C.}
{\nameref{RF:05}}
{El programa deberá ser capaz de ejecutar una a una las lineas de código de un programa en lenguaje C}
{El usuario deberá haber compilado el archivo}
{{1: & se ejecutará una linea de código \\
\hline
2: & Irá mostrando las variables en memoria de ese programa acorde al \nameref{RF:09}\\}
{1: & no se dispone de un ejecutable funcional: se vuelve al \nameref{RF:05} \\}}
{--}

\outerTable{08}{Ejecutar código completo}
{Versión 1.0}
{OBJ-01:Depuración de código C.}
{\nameref{RF:05}}
{El programa deberá ser capaz de ejecutar un programa en lenguaje C completo}
{El usuario deberá haber compilado el archivo}
{{1: & se ejecutará el programa completo \\
\hline
2: & se mostrá el estado final de todas la variables acorde al \nameref{RF:09}\\}
{1: & no se dispone de un ejecutable funcional: se vuelve al \nameref{RF:05}\\}}
{--}

\outerTable{09}{Mostrar variables en memoria}
{Versión 1.0}
{OBJ-01:Depuración de código C.}
{\item\nameref{RF:06}
\item\nameref{RF:07}
\item\nameref{RF:08}}
{El programa deberá ser capaz de mostrar en una tabla las variables en memoria un un programa C}
{El usuario deberá empezado a ejecutar un programa C}
{{1: & cada vez que se cree o modifique una variable se mostrará en la tabla \\}
{-- & --\\}}
{--}